%\documentclass[twocolumn, trackchanges]{aastex6}
\documentclass[twocolumn]{aastex6}


\bibliographystyle{aasjournal}
\usepackage{graphicx}
\usepackage[suffix=]{epstopdf}
\usepackage{natbib}
\usepackage{amsmath}
\usepackage{url}
\usepackage{xspace}



%    Make Scientific Notation
\providecommand{\e}[1]{\ensuremath{\times 10^{#1}}}

% make the word Kepler italicized, deal w/ floating space afterwards
\newcommand{\Kepler}{\textsl{Kepler}\xspace}

\begin{document}

%%%%%%%%%%%%%%%%%%%%%%
\title{Constraining Stellar Active Latitudes with Transiting Exoplanets}

\shorttitle{GALEX View of ``Boyajian's Star''}
\shortauthors{Davenport et al.}

\author{
	James R. A. Davenport\altaffilmark{1,2}
	Brett M. Morris\altaffilmark{2},
	Leslie Hebb\altaffilmark{3}, 
	Michelle Gomez\altaffilmark{3},
	{\bf order is mutable}
	%Suzanne L. Hawley\altaffilmark{4}
	}

\altaffiltext{1}{NSF Astronomy and Astrophysics Postdoctoral Fellow, DIRAC Fellow}
\altaffiltext{2}{Department of Astronomy, University of Washington, Box 351580, Seattle, WA 98195}
\altaffiltext{3}{Department of Physics, Hobart and William Smith Colleges, Geneva, NY, 14456}



%%%%%%%%%%%%%%%%%%%%%%%%%%%%%%
\begin{abstract}
Active latitude bands, as well as the ``Solar Butterfly Diagram'', are foundational properties that drive models of the solar dynamo. However, while considerable progress has been made on modeling stellar dynamos, no comparable constraint for the latitude distribution of active regions has been available. Here we present an ensemble approach to studying the latitude distribution of starspots using \Kepler transiting exoplanets. By exploiting the differences in impact parameter ($b$), the planets occult a range of latitude bands. We find X, using Y stars, and note Z. The future is bright.
\end{abstract}



%%%%%%%%%%%%%%%%%%%%%%%%%%%%%%
\section{Introduction}
\label{sec:intro}

\cite{morris2017}

\cite{gomez2015}

\cite{davenport_phd}


%%%%%%%%%%%%%%%%%%%%%%
\section{Short Timescale Variability}
\label{sec:short}


%%%%%
\begin{figure}[!t]
\centering
\includegraphics[width=3.5in]{archive.png}
\caption{{\bf [Replace]} Distribution of Impact Parameters versus }
\label{fig:medtime}
\end{figure}





%%%%%%%%%%%%%%%%%%%%%%
\section{Summary}
\label{sec:summary}


%%%%%%%%%%%%%%%%%
\acknowledgments

JRAD is supported by an NSF Astronomy and Astrophysics Postdoctoral Fellowship under award AST-1501418. 


%%%%%%%%%%%%%%%%%
\bibliography{/Users/james/Dropbox/references.bib}

\end{document}
