%\documentclass[twocolumn, trackchanges]{aastex6}
\documentclass[preprint2]{aastex61}


\bibliographystyle{aasjournal}
\usepackage{graphicx}
\usepackage[suffix=]{epstopdf}
\usepackage{natbib}
\usepackage{amsmath}
\usepackage{url}
\usepackage{xspace}



%    Make Scientific Notation
\providecommand{\e}[1]{\ensuremath{\times 10^{#1}}}

% make the word Kepler italicized, deal w/ floating space afterwards
\newcommand{\Kepler}{\textsl{Kepler}\xspace}

\begin{document}

%%%%%%%%%%%%%%%%%%%%%%
\title{Constraining Stellar Active Latitudes with Transiting Exoplanets}

\shorttitle{GALEX View of ``Boyajian's Star''}
\shortauthors{Davenport et al.}


\author{James. R. A. Davenport}
\affiliation{Department of Astronomy, University of Washington, Seattle, WA 98195, USA}
\affiliation{NSF Astronomy and Astrophysics Postdoctoral Fellow}

\author{Brett M. Morris}
\affiliation{Department of Astronomy, University of Washington, Seattle, WA 98195, USA}

\author{Leslie Hebb}
\affiliation{Department of Physics, Hobart and William Smith Colleges, Geneva, NY, 14456}

\author{Michelle Gomez}
\affiliation{Department of Physics, Hobart and William Smith Colleges, Geneva, NY, 14456}

\author{Eric Agol}
\affiliation{Department of Astronomy, University of Washington, Seattle, WA 98195, USA}

\author{Suzanne L. Hawley}
\affiliation{Department of Astronomy, University of Washington, Seattle, WA 98195, USA}



%%%%%%%%%%%%%%%%%%%%%%%%%%%%%%
\begin{abstract}
Active latitude bands, as well as the ``Solar Butterfly Diagram'', are foundational properties that drive models of the solar dynamo. However, while considerable progress has been made on modeling stellar dynamos, no comparable constraint for the latitude distribution of active regions has been available. Here we present an ensemble approach to studying the latitude distribution of starspots using \Kepler transiting exoplanets. By exploiting the differences in impact parameter ($b$), the planets occult a range of latitude bands. We find X, using Y stars, and note Z. The future is bright.
\end{abstract}



%%%%%%%%%%%%%%%%%%%%%%%%%%%%%%
\section{Introduction}
\label{sec:intro}

best example of this for Hat-P-11 using misaligned transits:
\cite{morris2017}, first pointed out by \cite{sanchis-ojeda2011}

\cite{roettenbacher2016} use interferrometry to directly image stellar surface, show no signs of solar-like spot distribution for $\zeta$ And.


idea discussed initially by:
\cite{gomez2015}

starspot modeling can accurately produce spot properties and evolution
\cite{davenport2015b}

we have used STSP to model transits of aligned systems, such as Kepler-17
\cite{davenport_phd}, where we note large in- versus out-of-transit scatter should be useful


%%%%%%%%%%%%%%%%%%%%%%
\section{Short Timescale Variability}
\label{sec:short}


%%%%%
\begin{figure}[!t]
\centering
\includegraphics[width=3.5in]{archive.png}
\caption{{\bf [Replace]} Distribution of Impact Parameters versus }
\label{fig:medtime}
\end{figure}





%%%%%%%%%%%%%%%%%%%%%%
\section{Summary}
\label{sec:summary}


%%%%%%%%%%%%%%%%%
\acknowledgments

JRAD is supported by an NSF Astronomy and Astrophysics Postdoctoral Fellowship under award AST-1501418. 


%%%%%%%%%%%%%%%%%
\bibliography{/Users/james/Dropbox/references.bib}

\end{document}
